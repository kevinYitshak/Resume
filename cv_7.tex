%%%%%%%%%%%%%%%%%%%%%%%%%%%%%%%%%%%%%%%%%
% "ModernCV" CV and Cover Letter
% LaTeX Template
% Version 1.3 (29/10/16)
%
% This template has been downloaded from:
% http://www.LaTeXTemplates.com
%
% Original author:
% Xavier Danaux (xdanaux@gmail.com) with modifications by:
% Vel (vel@latextemplates.com)
%
% License:
% CC BY-NC-SA 3.0 (http://creativecommons.org/licenses/by-nc-sa/3.0/)
%
% Important note:
% This template requires the moderncv.cls and .sty files to be in the same
% directory as this .tex file. These files provide the resume style and themes
% used for structuring the document.
%
%%%%%%%%%%%%%%%%%%%%%%%%%%%%%%%%%%%%%%%%%

%----------------------------------------------------------------------------------------
%	PACKAGES AND OTHER DOCUMENT CONFIGURATIONS
%----------------------------------------------------------------------------------------

\documentclass[11pt,a4paper,sans]{moderncv} % Font sizes: 10, 11, or 12; paper sizes: a4paper, letterpaper, a5paper, legalpaper, executivepaper or landscape; font families: sans or roman

\moderncvstyle{classic} % CV theme - options include: 'casual' (default), 'classic', 'oldstyle' and 'banking'
\moderncvcolor{blue} % CV color - options include: 'blue' (default), 'orange', 'green', 'red', 'purple', 'grey' and 'black'

% \usepackage{lipsum} % Used for inserting dummy 'Lorem ipsum' text into the template

\usepackage[scale=0.85]{geometry} % Reduce document margins
%\setlength{\hintscolumnwidth}{3cm} % Uncomment to change the width of the dates column
%\setlength{\makecvtitlenamewidth}{10cm} % For the 'classic' style, uncomment to adjust the width of the space allocated to your name

%----------------------------------------------------------------------------------------
%	NAME AND CONTACT INFORMATION SECTION
%----------------------------------------------------------------------------------------

\firstname{Kevin} % Your first name
\familyname{Raj} % Your last name

% All information in this block is optional, comment out any lines you don't need
\title{Research Assistant}
% \address{123 Broadway}{City, State 12345}
% \mobile{(000) 111 1111}
% \phone{(000) 111 1112}
% \fax{(000) 111 1113}
\email{kevinyitshak@gmail.com}
\homepage{kevinyitshak.github.io}{kevinyitshak.github.io} % The first argument is the url for the clickable link, the second argument is the url displayed in the template - this allows special characters to be displayed such as the tilde in this example
% \extrainfo{additional information}
% \photo[70pt][0.4pt]{pictures/picture} % The first bracket is the picture height, the second is the thickness of the frame around the picture (0pt for no frame)
% \quote{"A witty and playful quotation" - John Smith}

%----------------------------------------------------------------------------------------

\begin{document}
%----------------------------------------------------------------------------------------
%	CURRICULUM VITAE
%----------------------------------------------------------------------------------------

\makecvtitle % Print the CV title

%----------------------------------------------------------------------------------------
%	EDUCATION SECTION
%----------------------------------------------------------------------------------------

\section{Education}

\cventry{2020--2022} {Masters in Electrical Engineering}
					{The Delft University of Technology}{Delft}
					{}
					{\textit {Specialization:} Signals \& Systems track.}

\cventry{2015--2019} {Bachelor on Electrical \& Electronics Engineering}
					{Manipal Institute of Technology}{Manipal}
					{}
					{\textit {Minor specialization:} Signals \& Systems.}  % Arguments not required can be left empty

\section{Bachelors Thesis}

\cvitem{Title}{Optic Disc Segmentation Using Modified Deep Retinal Image Understanding (DRIU).}
\cvitem{Supervisors}{Dr. Chandra Sekhar Seelamantula \& Assistant Professor Harish Kumar J.R.}
\cvitem{Description}{Proposed a technique for Optic Disc segmentation using retinal fundus
					images by combining squeeze \& excite layer with Resnet-50 architecture,
					which acts as an initial point for Glaucoma assessment.}

%----------------------------------------------------------------------------------------
%	INTERESTS SECTION
%----------------------------------------------------------------------------------------

\section{Interests}

% \renewcommand{\listitemsymbol}{-~} % Changes the symbol used for lists
\cvitem{}{Image Processing, Signal Processing, Computer Vision, Machine Learning, Linear Algebra.}

%----------------------------------------------------------------------------------------
%	WORK EXPERIENCE SECTION
%----------------------------------------------------------------------------------------

\section{Experience}

\cventry{Aug'19--Present}{Research Assistant}{\textsc{Spectrum Lab}}{IISc}
		{}
		{Developing a complete pipeline for the analysis of Wireless Capsule Endoscopy (WCE)
			images, in collaboration with \href{https://qpiai.tech/}{\textsc{QpiAI\textsuperscript{TM}}.}
		and Indian Airforce command hospital, Bangalore.\textit{Tool used:} Pytorch.}

%------------------------------------------------
\cventry{Jan'19-June'19}{Research Intern}{\textsc{Spectrum Lab}}{IISc}
		{}
		{Proposed an Artery-Vein classification network from single-wavelength fundus
		images using the low-level to high-level features extracted from Identity mapping network,
	which acts as a backbone architecture. I also developed an ImageJ plugin and android
application based on the ‘ICIP 2019’ paper.
		\textit{Tool used:} keras, ImageJ, Java.}

%------------------------------------------------

\cventry{May'18-July'18}{Summer Research Intern}{\textsc{Spectrum Lab}}{IISc}
		{}
		{Proposed a novel methodology using a multi-scale Harris corner technique and
		iterative Voronoi decomposition technique for optic cup segmentation using the structural
	properties of blood vessels. The Ministry of Human Resource Development (MHRD) ,India,
under the IMPRINT initiative, funded this project. \textit{Tool used:} MATLAB.}
%------------------------------------------------

%----------------------------------------------------------------------------------------
%	PUBLICATIONS
%----------------------------------------------------------------------------------------

\section{Publications}

\pubentry{2020}{\textbf {P. Kevin Raj}, Aniketh Manjunath, J.R.H. Kumar and Chandra S. Seelamantula}
				{``Automatic Classification of Artery-Vein from Single Wavelength Fundus Images''}
				{In Proc. IEEE International Symposium on Biomedical Imaging (ISBI)}
				{Iowa, USA, 2020}{\href{https://ieeexplore.ieee.org/abstract/document/9098580}{\textcolor{blue}{[pdf]}}}

\pubentry{2019}{\textbf{P. Kevin Raj}, J.R.H Kumar, S. Jois, S. Harsha and Chandra S. Seelamantula}
				{``A Structure Tensor based Voronoi Decomposition Technique for Optic Cup Segmentation''}
				{In Proc. IEEE International Conference on Image Processing (ICIP)}
				{Taipei, Taiwan, 2019}
				{\href{https://ieeexplore.ieee.org/document/8804286}{\textcolor{blue}{[pdf]}}
				\href{https://1drv.ms/b/s!AujsDtKxFNJCg3Fv4Gg8icPY7mMJ?e=13q3mt}{\textcolor{red}{[Oral]}}}

\pubentry{2019}{J.R.H. Kumar, K. Teotia, \textbf{P. Kevin Raj}, A. Jasbon, K.V. Rajagopal and Chandra S. Seelamantula}
				{``Automatic Segmentation of Common Carotid Artery in Longitudinal Mode Ultrasound Images Using Active Oblongs''}
				{In Proc. IEEE International Conference on Acoustics, Speech and Signal Processing (ICASSP)}
				{Brighton, UK, 2019}
				{\href{https://ieeexplore.ieee.org/document/8682301}{\textcolor{blue}{[pdf]}}}

%----------------------------------------------------------------------------------------
%	AWARDS SECTION
%----------------------------------------------------------------------------------------

\section{Awards}

\cvitem{2019}{\textbf{Travel grant:} Amount of 940\$ awarded by IEEE Signal Processing Society to attend ICIP'19}

%----------------------------------------------------------------------------------------
%	COMPUTER SKILLS SECTION
%----------------------------------------------------------------------------------------

\section{Skills \& Certifications}

\cvitem{Skills}{Python, \textsc{html}, \LaTeX, Pytorch, Keras, MATLAB}
\cvitem{Certifications}{Image and Video Processing, by Duke \textbf{Coursera}, Digital and Signal Processing, by EPFL \textbf{Coursera} , Neural Networks and Deep Learning, Machine Learning, and Hyperparameter tuning, Regularization and Optimization, by Deeplearning.ai \textbf{Coursera}.}
\end{document}
